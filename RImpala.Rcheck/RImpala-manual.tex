\nonstopmode{}
\documentclass[letterpaper]{book}
\usepackage[times,inconsolata,hyper]{Rd}
\usepackage{makeidx}
\usepackage[utf8,latin1]{inputenc}
% \usepackage{graphicx} % @USE GRAPHICX@
\makeindex{}
\begin{document}
\chapter*{}
\begin{center}
{\textbf{\huge Package `RImpala'}}
\par\bigskip{\large \today}
\end{center}
\begin{description}
\raggedright{}
\item[Version]\AsIs{0.1.4}
\item[Date]\AsIs{2014-09-04}
\item[Title]\AsIs{RImpala}
\item[Author]\AsIs{Vijay Raajaa, Austin Chungath Vincent, Sachin Sudarshana, Vikas Raguttahalli}
\item[Maintainer]\AsIs{Vijay Raajaa }\email{vijay.raajaa@mu-sigma.com}\AsIs{}
\item[Contact]\AsIs{Austin Chungath Vincent }\email{austincv@gmail.com}\AsIs{,Vikas
Raguttahalli }\email{vikas.r@mu-sigma.com}\AsIs{, Sachin Sudarshana
}\email{sachin.sudarshana@gmail.com}\AsIs{}
\item[Description]\AsIs{RImpala is a package that helps you to connect and execute distributed queries using Cloudera Impala, which is a massively parallel processing (MPP) SQL query engine that runs natively in Apache Hadoop. Impala supports jdbc integration which RImpala utilizes to establish a connection between R and Impala.}
\item[Depends]\AsIs{R (>= 2.7.0), rJava (>= 0.5-0)}
\item[SystemRequirements]\AsIs{Java (>= 1.5)}
\item[License]\AsIs{GPL-3}
\end{description}
\Rdcontents{\R{} topics documented:}
\inputencoding{utf8}
\HeaderA{RImpala-package}{A Package to interface R and Impala}{RImpala.Rdash.package}
\keyword{Impala}{RImpala-package}
\keyword{Installation}{RImpala-package}
\keyword{JDBC}{RImpala-package}
\keyword{drivers}{RImpala-package}
%
\begin{Description}\relax
RImpala-package contains the R functions required to connect, execute queries and retrieve back results from Impala. It uses the rJava package to create a JDBC connection to any of the impala servers running on a Hadoop Cluster. 
\end{Description}
%
\begin{Details}\relax

\Tabular{ll}{
Package: & RImpala\\{}
Type: & Package\\{}
Version: & 1.0.0\\{}
Date: & 2013-09-06\\{}
License: & file LICENSE\\{}
}
\end{Details}
%
\begin{Section}{Installation}
RImpala uses the JDBC drivers provided by Cloudera Impala. We need to install them before we can use the RImpala package. Cloudera provides the JBDC jars on their website that can be downloaded directly.\\{}
\\{}
There are two ways to do this:\\{}
\\{}
A. If you have Cloudera Impala installed on the machine running R then you will have the necessary JDBC jars already in place (probably in "/usr/lib/impala/lib") and you can use them to initiate the connection to Impala.\\{}
B. If the machine running R is a different server than the Impala server then you need to download the JDBC jars from \bold{https://downloads.cloudera.com/impala-jdbc/impala-jdbc-0.5-2.zip} or from the server running Impala and extract it to a location that can be accessed by the R user.\\{}
\\{}
After you have installed the JDBC drivers you can start using the RImpala package.\\{}
Have a look at \code{\LinkA{rimpala.init}{rimpala.init}} and \code{\LinkA{rimpala.connect}{rimpala.connect}} to establish connection to Impala.\\{}
\end{Section}
%
\begin{Author}\relax
Austin Chungath Vincent \email{austin.cv@mu-sigma.com},\\{}
Vikas Raguttahalli \email{vikas.r@mu-sigma.com},\\{}
Sachin Sudarshana \email{sachin.sudarshana@mu-sigma.com}
\end{Author}
%
\begin{References}\relax
http://www.cloudera.com/content/cloudera/en/products/cdh/impala.html - Cloudera's page on Impala
\end{References}
\inputencoding{utf8}
\HeaderA{rimpala.close}{Function to close the JDBC connection to Impala}{rimpala.close}
\keyword{close}{rimpala.close}
\keyword{\textbackslash{}textasciitilde{}kwd2}{rimpala.close}
%
\begin{Description}\relax
This function closes a sucessful connection to Impala-server
\end{Description}
%
\begin{Usage}
\begin{verbatim}
rimpala.close()
\end{verbatim}
\end{Usage}
%
\begin{Value}
"Connection Closed" is displayed on the console when the JDBC connection is successfully closed
\end{Value}
%
\begin{Author}\relax
Austin Chungath Vincent \email{austin.cv@mu-sigma.com},\\{}
Vikas Raguttahalli \email{vikas.r@mu-sigma.com},\\{}
Sachin Sudarshana \email{sachin.sudarshana@mu-sigma.com}
\end{Author}
%
\begin{Examples}
\begin{ExampleCode}
## Not run: 
library(RImpala)
rimpala.init()
rimpala.connect(IP="127.0.0.1",port="21050")
rimpala.close()

## End(Not run)
\end{ExampleCode}
\end{Examples}
\inputencoding{utf8}
\HeaderA{rimpala.connect}{ Establishes a JDBC connection to a machine running Impala}{rimpala.connect}
\keyword{connection}{rimpala.connect}
\keyword{\textbackslash{}textasciitilde{}kwd2}{rimpala.connect}
%
\begin{Description}\relax
This function creates a connection to the impalad daemon running on a machine in a Hadoop Cluster. The IP of the machine and the port on which the impalad daemon is running is passed as an argument.
\end{Description}
%
\begin{Usage}
\begin{verbatim}
rimpala.connect(IP="localhost",port="21050",principal="noSasl")
\end{verbatim}
\end{Usage}
%
\begin{Arguments}
\begin{ldescription}
\item[\code{IP}] 
The IP of the machine to which the connection needs to be established. Default value is localhost

\item[\code{port}] 
The port on the machine where the Impala daemon is running. Default value is 21050

\item[\code{principal}] 
The principal to use if you require Kerberos authentication.The principal must be the same user principal you used when starting Impala. For example: "impala/myhost.example.com@H2.EXAMPLE.COM". Default value is "noSasl"

\end{ldescription}
\end{Arguments}
%
\begin{Value}
"Connection Established" is displayed on the console upon successful connection. 
\end{Value}
%
\begin{Author}\relax
Austin Chungath Vincent \email{austin.cv@mu-sigma.com},\\{}
Vikas Raguttahalli \email{vikas.r@mu-sigma.com},\\{}
Sachin Sudarshana \email{sachin.sudarshana@mu-sigma.com}
\end{Author}
%
\begin{Examples}
\begin{ExampleCode}
## Not run: 
library("RImpala")
rimpala.init()
rimpala.connect(IP="127.0.0.1",port="21050")
rimpala.close()
rimpala.connect(IP="localhost",port="21050",principal="impala/myhost.example.com@H2.EXAMPLE.COM")

## End(Not run)
\end{ExampleCode}
\end{Examples}
\inputencoding{utf8}
\HeaderA{rimpala.describe}{Function to describe any table present in Hive's metastore}{rimpala.describe}
\keyword{describe}{rimpala.describe}
\keyword{\textbackslash{}textasciitilde{}kwd2}{rimpala.describe}
%
\begin{Description}\relax
This function runs the describe query of Impala against the table passed as an argument to the function
\end{Description}
%
\begin{Usage}
\begin{verbatim}
rimpala.describe(table)
\end{verbatim}
\end{Usage}
%
\begin{Arguments}
\begin{ldescription}
\item[\code{table}] 
The name of the table that needs to be described

\end{ldescription}
\end{Arguments}
%
\begin{Value}
Returns an dataframe that contains the details of the table as displayed by the describe command
\end{Value}
%
\begin{Author}\relax
Austin Chungath Vincent \email{austin.cv@mu-sigma.com},\\{}
Vikas Raguttahalli \email{vikas.r@mu-sigma.com},\\{}
Sachin Sudarshana \email{sachin.sudarshana@mu-sigma.com}
\end{Author}
%
\begin{Examples}
\begin{ExampleCode}
## Not run: 
library("RImpala")
rimpala.init()
rimpala.connect("127.0.0.1","21050")
des=rimpala.describe(table="sample_table")

## End(Not run)
\end{ExampleCode}
\end{Examples}
\inputencoding{utf8}
\HeaderA{rimpala.init}{ Adds the folder containing the jars for Impala in the  Classpath}{rimpala.init}
\keyword{connection}{rimpala.init}
\keyword{\textbackslash{}textasciitilde{}kwd2}{rimpala.init}
%
\begin{Description}\relax
Initializing the package by adding the required jars to the Classpath
\end{Description}
%
\begin{Usage}
\begin{verbatim}
rimpala.init(impala_home=NULL,libs="/usr/lib/impala/lib")
\end{verbatim}
\end{Usage}
%
\begin{Arguments}
\begin{ldescription}
\item[\code{impala\_home}] 
The home folder of Impala. Default is NULL

\item[\code{libs}] 
The directory in which the jars required for establishing a connection to Impala are required
Default path is "/usr/lib/impala/lib"
\end{ldescription}
\end{Arguments}
%
\begin{Details}\relax
This should be the first function that should be executed once the RImpala package is installed and loaded 
\end{Details}
%
\begin{Value}
"Classpath added succesfully" is displayed on the addition of a valid path.
\end{Value}
%
\begin{Author}\relax
Austin Chungath Vincent \email{austin.cv@mu-sigma.com},\\{}
Vikas Raguttahalli \email{vikas.r@mu-sigma.com},\\{}
Sachin Sudarshana \email{sachin.sudarshana@mu-sigma.com}
\end{Author}
%
\begin{Examples}
\begin{ExampleCode}
## Not run: 
library("RImpala")
rimpala.init(libs="/usr/lib/impala/lib")

## End(Not run)
\end{ExampleCode}
\end{Examples}
\inputencoding{utf8}
\HeaderA{rimpala.invalidate}{ Invalidates the metadata of a one or all tables}{rimpala.invalidate}
\keyword{invalidate}{rimpala.invalidate}
\keyword{\textbackslash{}textasciitilde{}kwd2}{rimpala.invalidate}
%
\begin{Description}\relax
This function invalidates metadata of the table passed as an argument to it. Metadata invalidation is required if a table has been changed in Hive.
\end{Description}
%
\begin{Usage}
\begin{verbatim}
rimpala.invalidate(table=" ")
\end{verbatim}
\end{Usage}
%
\begin{Arguments}
\begin{ldescription}
\item[\code{table}] 
The name of the table whose metadata needs to be invalidated. Default is NULL

\end{ldescription}
\end{Arguments}
%
\begin{Value}
The metadata of the table passed as an argumented is invalidated or marked as stale from the cache. If no argument is passed, all the metadata of all the tables are invalidated.
\end{Value}
%
\begin{Author}\relax
Austin Chungath Vincent \email{austin.cv@mu-sigma.com},\\{}
Vikas Raguttahalli \email{vikas.r@mu-sigma.com},\\{}
Sachin Sudarshana \email{sachin.sudarshana@mu-sigma.com}
\end{Author}
%
\begin{Examples}
\begin{ExampleCode}
## Not run: 
library("RImpala")
rimpala.init()
rimpala.connect(IP="127.0.0.1",port="21050")
rimpala.invalidate(table="sample")

## End(Not run)
\end{ExampleCode}
\end{Examples}
\inputencoding{utf8}
\HeaderA{rimpala.query}{Function to run a Query in Impala}{rimpala.query}
\keyword{\textbackslash{}textasciitilde{}kwd1}{rimpala.query}
\keyword{\textbackslash{}textasciitilde{}kwd2}{rimpala.query}
%
\begin{Description}\relax
This function executes the Query specified as an argument in Impala. If no query is passed, the show tables query is run as default
\end{Description}
%
\begin{Usage}
\begin{verbatim}
rimpala.query(Q="show tables")
\end{verbatim}
\end{Usage}
%
\begin{Arguments}
\begin{ldescription}
\item[\code{Q}] 
The Query to be executed on Impala. The default query is show tables.

\end{ldescription}
\end{Arguments}
%
\begin{Value}
The result of the Query is returned into a dataframe if the Query is valid and does not have any errors.
\end{Value}
%
\begin{Author}\relax
Austin Chungath Vincent \email{austin.cv@mu-sigma.com},\\{}
Vikas Raguttahalli \email{vikas.r@mu-sigma.com},\\{}
Sachin Sudarshana \email{sachin.sudarshana@mu-sigma.com}
\end{Author}
%
\begin{Examples}
\begin{ExampleCode}
## Not run: 
library("RImpala")
rimpala.init()
rimpala.connect(IP="127.0.0.1",port="21050")
res = rimpala.query("Select * from sample_table")

## End(Not run)
\end{ExampleCode}
\end{Examples}
\inputencoding{utf8}
\HeaderA{rimpala.refresh}{ Refreshes and loads the new metadata for the given table}{rimpala.refresh}
\keyword{refresh}{rimpala.refresh}
\keyword{\textbackslash{}textasciitilde{}kwd2}{rimpala.refresh}
%
\begin{Description}\relax
This function refreshes the metadata of the table passed as an argument to it.
\end{Description}
%
\begin{Usage}
\begin{verbatim}
rimpala.refresh(table="table_name")
\end{verbatim}
\end{Usage}
%
\begin{Arguments}
\begin{ldescription}
\item[\code{table}] 
The name of the table whose metadata needs to be refreshed. This is a mandatory argument.

\end{ldescription}
\end{Arguments}
%
\begin{Value}
The metadata of the table passed as an argument is refreshed and the new metadata is immediately loaded into the cache.
\end{Value}
%
\begin{Author}\relax
Austin Chungath Vincent \email{austin.cv@mu-sigma.com},\\{}
Vikas Raguttahalli \email{vikas.r@mu-sigma.com},\\{}
Sachin Sudarshana \email{sachin.sudarshana@mu-sigma.com}
\end{Author}
%
\begin{Examples}
\begin{ExampleCode}
## Not run: 
library("RImpala")
rimpala.init()
rimpala.connect(IP="127.0.0.1",port="21050")
rimpala.refresh(table="sample")

## End(Not run)
\end{ExampleCode}
\end{Examples}
\inputencoding{utf8}
\HeaderA{rimpala.showdatabases}{Function to list all the databases present }{rimpala.showdatabases}
\keyword{show databases}{rimpala.showdatabases}
\keyword{\textbackslash{}textasciitilde{}kwd2}{rimpala.showdatabases}
%
\begin{Description}\relax
This function returns the list of databases present in Hive's metastore that is leveraged by Impala
\end{Description}
%
\begin{Usage}
\begin{verbatim}
rimpala.showdatabases()
\end{verbatim}
\end{Usage}
%
\begin{Value}
The list of databases present in Hive's metastore is returned into a dataframe.
\end{Value}
%
\begin{Author}\relax
Austin Chungath Vincent \email{austin.cv@mu-sigma.com},\\{}
Vikas Raguttahalli \email{vikas.r@mu-sigma.com},\\{}
Sachin Sudarshana \email{sachin.sudarshana@mu-sigma.com}
\end{Author}
%
\begin{Examples}
\begin{ExampleCode}
## Not run: 
library("RImpala")
rimpala.init()
rimpala.connect("127.0.0.1","21050")
rimpala.showdatabases()

## End(Not run)
\end{ExampleCode}
\end{Examples}
\inputencoding{utf8}
\HeaderA{rimpala.showtables}{Function to display the list of all the tables present}{rimpala.showtables}
\keyword{show tables}{rimpala.showtables}
\keyword{\textbackslash{}textasciitilde{}kwd2}{rimpala.showtables}
%
\begin{Description}\relax
This function retrieves the list of tables present in the current working database
\end{Description}
%
\begin{Usage}
\begin{verbatim}
rimpala.showtables()
\end{verbatim}
\end{Usage}
%
\begin{Value}
List of tables present in the current database is returned into a dataframe
\end{Value}
%
\begin{Author}\relax
Austin Chungath Vincent \email{austin.cv@mu-sigma.com},\\{}
Vikas Raguttahalli \email{vikas.r@mu-sigma.com},\\{}
Sachin Sudarshana \email{sachin.sudarshana@mu-sigma.com}
\end{Author}
%
\begin{Examples}
\begin{ExampleCode}
## Not run: 
library("RImpala")
rimpala.init()
rimpala.connect(IP="127.0.0.1",port="21050")
rimpala.showtables()

## End(Not run)
\end{ExampleCode}
\end{Examples}
\inputencoding{utf8}
\HeaderA{rimpala.usedatabase}{Function to change the current working database}{rimpala.usedatabase}
\keyword{use database}{rimpala.usedatabase}
\keyword{\textbackslash{}textasciitilde{}kwd2}{rimpala.usedatabase}
%
\begin{Description}\relax
This function changes the current database to the database specified as an argument to the function
\end{Description}
%
\begin{Usage}
\begin{verbatim}
rimpala.usedatabase(db)
\end{verbatim}
\end{Usage}
%
\begin{Arguments}
\begin{ldescription}
\item[\code{db}] 
The name of the database.

\end{ldescription}
\end{Arguments}
%
\begin{Value}
Changes the database to the specified database and prints "Database changed to \emph{<Database name>}" on the console
\end{Value}
%
\begin{Author}\relax
Austin Chungath Vincent \email{austin.cv@mu-sigma.com},\\{}
Vikas Raguttahalli \email{vikas.r@mu-sigma.com},\\{}
Sachin Sudarshana \email{sachin.sudarshana@mu-sigma.com}
\end{Author}
%
\begin{Examples}
\begin{ExampleCode}
## Not run: 
library("RImpala")
rimpala.init()
rimpala.connect(127.0.0.1,"21050")
rimpala.usedatabase(db="sample_db")

## End(Not run)
\end{ExampleCode}
\end{Examples}
\printindex{}
\end{document}
